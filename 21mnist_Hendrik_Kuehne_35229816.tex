\documentclass{article}

\input{latexauxfiles/packages.tex}
\input{latexauxfiles/commands.tex}

\numberwithin{equation}{section}
\numberwithin{figure}{section}
\numberwithin{table}{section}
\addbibresource{./latexauxfiles/references.bib}

\begin{document}
    \begin{center}
        Hendrik Kühne - 35229816
    \end{center}

    \tableofcontents
    \vspace{\baselineskip}
    \vspace{\baselineskip}

    In the following, we will go into detail on how the code for a neural network from Ref.~\cite{Taura:parallel-distributed}
    can be accelerated. The corresponding code can be found on GitHub \cite{Kühne:parallel-distributed}. After introducing
    the modifications on data structures, we will show how parts of the neural network were accelerated and compare the
    results.

    We used SIMD to accelerate the forward and backward passes of convolutional and linear layers on two
    hardware platforms: x86 and Arm.

    \section{Vectorizing the \texorpdfstring{\mintinline{c++}{tensor}}{tensor} struct}
    \label{sec:tensor}

    In order to parallelize the neural network, we need a way to extract vectors from the \mintinline{c++}{tensor} instances
    that form the backbone of the network. We use different vector types, depending on the hardware platform. We choose
    those while keeping in mind that the default numerical type in Ref.~\cite{Taura:parallel-distributed} is
    \mintinline{c++}{float}, which is a 32-Bit type. For achieving a high degree of parallelism, we want vectors with many
    lanes.

    \begin{itemize}
        \item On Arm, there are 64-Bit (D-type) und 128-Bit (Q-type) vectors \cite{NEON:VectorDataTypes}. The best compromise
        between the necessary accuracy and high parallelism is the \mintinline{c++}{float32x4_t} Q-type, which has four lanes
        with 32-Bit precision.
        \item On x86, we define a datatype \mintinline{c++}{realv} with
        
        \begin{minted}[linenos]{c++}
            typedef float real;
            typedef real realv __attribute__((vector_size(64),
                                              __may_alias__,
                                              aligned(sizeof(real))));
        \end{minted}

        which is a vector that comprises 16 32-Bit floating point values. This definition is added to the file
        \texttt{tensor.h}.
    \end{itemize}

    Using the appropriate preprocessor directive, the necessary headers for the corresponding intrinsics are included with

    \begin{minted}[linenos]{c++}
        #ifdef __ARM_64BIT_STATE
            #include<arm_neon.h>
        #else
            #include<x86intrin.h>
        #endif
    \end{minted}

    where the variable \mintinline{c++}{__ARM_64BIT_STATE} is only defined on Arm-hardware. The headers \texttt{arm\_neon.h}
    respectively \texttt{x86intrin.h} contain the vector instructions for Arm \cite{NEON:intrinsics} or x86
    \cite{Intel:intrinsics} platforms.

    We would now like to extract a vector from a \mintinline{c++}{tensor} instance. We do this by casting a pointer to the
    memory to be vectorized as the corresponding vector type pointer and then returning the de-referenced pointer. To this
    end, we add two functions per architecture to \texttt{tensor.h}; for Arm, these are

    \begin{minted}[linenos]{c++}
        template<typename T>
        static float32x4_t& V4(T& address){return *((float32x4_t*)address);}

        float32x4_t& tensor<T,N0,N1,N2,N3>::V4(idx_t i0,
                                               idx_t i1=0,
                                               idx_t i2=0,
                                               idx_t i3=0){
            tensor<T,N0,N1,N2,N3>& a = *this;
            T* address = &(a(i0,i1,i2,i3));

            return ::V4(address);
        }
    \end{minted}

    where \mintinline{c++}{idx_t} is an integer type.

    The function \mintinline{c++}{tensor<T,N0,N1,N2,N3>::V4} returns a pointer to the first value of the desired vector
    by reference and \mintinline{c++}{V4} then returns a vector with four lanes. These functions are analogously named
    \mintinline{c++}{tensor<T,N0,N1,N2,N3>::V16} and \mintinline{c++}{V16} on x86; they return a 16-lane vector using the
    same mechanism. These vectors are write-safe, meaning values can be assigned to them.
    
    The default arguments in the function call \mintinline{c++}{tensor<T,N0,N1,N2,N3>::V4} /
    \mintinline{c++}{tensor<T,N0,N1,N2,N3>::V16} serve an important purpose: Since we are casting a pointer to obtain a
    SIMD-vector, we are assuming the data to be saved consecutively in memory. In general, we could thus only extract vectors
    from the last dimension of a tensor since only these values are saved consecutively. In the special case that, from a
    certain dimension onward, all dimensions of the tensor have size 1, this is not true. Consider for example the instance
    \mintinline{c++}{tensor<real,5,5,1,1> a}. Since the last two dimensions have size one, the memory layout is actually
    consecutive in dimension two. The default arguments \mintinline{c++}{idx_t i2 = 0} and \mintinline{c++}{idx_t i3 = 0}
    then ensure that \mintinline{c++}{a.V4(2,0)} returns a vector containing the elements \mintinline{c++}{a(2,0:4,0,0)}.
    
    Using the above definitions for vector extraction from \mintinline{c++}{tensor} instances thus enables us to extract
    vectors from the last dimension with size $\mintinline{c++}{Ni} > 1$.
    
    \section{Vectorizing the Convolutional and Linear Layers}
    \label{sec:layers}

    As mentioned earlier, we are vectorizing the forward and backward passes in \texttt{convolution.h} and \texttt{linear.h}.
    To this end, we add the implementation \mintinline{c++}{algo_cpu_simd} which is set as current implementation with the
    command line argument \texttt{-a cpu\_simd}. During computation, the functions

    \begin{minted}[linenos]{c++}
        void Convolution2D<maxB,IC,H,W,K,OC>::forward_simd(
                    tensor<real,maxB,IC,H,W>& x, int training);
        void Convolution2D<maxB,IC,H,W,K,OC>::backward_simd(
                    tensor<real,maxB,OC,H-K+1,W-K+1>& gy);
        void Linear<M,N,K0,K1,K2>::forward_simd(
                    tensor<real,M,K0,K1,K2>& x, int training);
        void Linear<M,N,K0,K1,K2>::backward_simd(
                    tensor<real,M,N>& gy);
    \end{minted}

    are called; these are the vectorized versions of their \texttt{base} counterparts. The code for both Arm and SIMD is
    contained in these functions; the preprocessor again chooses the right one based on the variable
    \mintinline{c++}{__ARM_64BIT_STATE}. A code excerpt from the forward pass in linear layers is discussed in
    Figure~\ref{layers:code:forward_simd}.

    \begin{figure}[htb]
        \begin{minted}[linenos]{c++}
            float32x4_t vec;
            real v;
            for(idx_t i = 0;i < m;i++){
                idx_t j = 0;
                for(;j+3 < N;j+=4){
                     /* We will parallelize the loop over j since we only
                     have access to vectors taken from the last dimension
                     of a tensor. We'll use vectors with four lanes. */
                    vec = vdupq_n_f32(0);
                    for(idx_t k0 = 0;k0 < K0;k0++){
                        for(idx_t k1 = 0;k1 < K1;k1++){
                            for(idx_t k2 = 0;k2 < K2;k2++){
                                // v += x(i,k0,k1,k2) * w(k0,k1,k2,j);
                                vec = vfmaq_f32(vec,
                                                w.V4(k0,k1,k2,j),
                                                vdupq_n_f32(x(i,k0,k1,k2)));
                                    // vfma(a,b,c) = a + b * c
                            }
                        }
                    }
                    vec = vaddq_f32(vec,b.V4(j));
                    y.V4(i,j) = vec;
                    // y(i,j) = v + b(j);
                }
                for(;j < N;j++){
                     /* remainder iterations - this is just the code from
                     forward_base. */
                    v = 0;
                    for(idx_t k0 = 0;k0 < K0;k0++){
                        for(idx_t k1 = 0;k1 < K1;k1++){
                            for(idx_t k2 = 0;k2 < K2;k2++){
                                v += x(i,k0,k1,k2) * w(k0,k1,k2,j);
                            }
                        }
                    }
                    y(i,j) = v + b(j);
                }
            }
        \end{minted}
        \caption{Arm code from \mintinline{c++}{void Linear::forward_simd}. As discussed in Section~\ref{sec:tensor}, we can
        only extract vectors from the last dimension of a tensor with size $\mintinline{c++}{Ni} > 1$, there are thus two
        loops that are candidates for vectorization: The ones over \mintinline{c++}{k2} or \mintinline{c++}{j}, respectively.
        We choose the one over \mintinline{c++}{j} since the range $0 \leq \mintinline{c++}{k2} < 3$ is smaller than our
        vector size. The function \mintinline{c++}{float32x4_t vdupq_n_f32(float32_t value)} then writes \mintinline{c++}{0}
        to all lanes of \mintinline{c++}{vec}. \mintinline{c++}{float32x4_t vfmaq_f32(float32x4_t a, float32x4_t b, float32x4_t c)}
        performs the fused multiply-add operation \mintinline{c++}{a = a + b * c} and \mintinline{c++}{vaddq_f32} performs
        addition. We finally write the result back to the tensor using the custom function \mintinline{c++}{tensor::V4}. Since
        the relevant tensor dimension is not necessarily divisible by four, we execute remainder iterations afterwards
        using the original code.}
        \label{layers:code:forward_simd}
    \end{figure}

    The other functions are vectorized in the same fashion: we identify the loop to be vectorized (one which iterates over the
    last dimension of a tensor), re-define variables as vectors as necessary and replace the operations with the appropriate
    intrinsics. An overview of the vector operations we used and their corresponding intrinsics can be found in
    Table~\ref{layers:tab:intrinsics}.

    \begin{table}[htb]
        \begin{equation*}
            \begin{array}{c|c|c}
                \text{Operation} & \text{Arm} & \text{x86} \\ \hline
                v_i = a & \mintinline{c++}{vdupq_n_f32(a)} & \mintinline{c++}{_mm512_set1_ps(a)} \\
                v_i \rightarrow v_i + a_ib_i & \mintinline{c++}{vfmaq_f32(v,a,b)} & \mintinline{c++}{_mm512_fmadd_ps(a,b,v)} \\
                v_i \rightarrow v_i + a_i & \mintinline{c++}{vaddq_f32(v,a)} & \mintinline{c++}{_mm512_add_ps(v,a)} \\
                \sum_iv_i & \mintinline{c++}{vaddvq_f32(v)} & \mintinline{c++}{_mm512_reduce_add_ps(v)}
            \end{array}
        \end{equation*}
        \caption{An overview of the vector operations we used and the corresponding intrinsics on Arm or x86
        \cite{NEON:intrinsics,Intel:intrinsics}. All of them accept and return \mintinline{c++}{float32x4_t} respectively
        \mintinline{c++}{__m512} types.}
        \label{layers:tab:intrinsics}
    \end{table}

    \FloatBarrier
    \section{Results}
    \label{sec:results}

    We first need to check that the results are indeed the same. We do this by training a network uing just one training
    data point (one image from the mnist dataset) for \num{10} epochs. The results are displayed in
    figures \ref{results:fig:cpu_base} and \ref{results:fig:cpu_simd}.

    \begin{figure}[htb]
        \inputminted{shell-session}{21mnist/cmd_output/Arm_cpu_base.txt}
        \caption{The command-line output for our tests using the option \mintinline{shell-session}{-a cpu_base}.}
        \label{results:fig:cpu_base}
    \end{figure}

    \begin{figure}[htb]
        \inputminted{shell-session}{21mnist/cmd_output/Arm_cpu_simd.txt}
        \caption{The command-line output for our tests using the option \mintinline{shell-session}{-a cpu_simd}.}
        \label{results:fig:cpu_simd}
    \end{figure}

    We see that the loss values are indeed the same for all epochs, so our implementation is correct.

    \begin{table}[htb]
        \begin{center}
            \begin{tabular}{c|cccc}
                & \multicolumn{2}{c}{Arm} & \multicolumn{2}{c}{x86} \\ \hline
                & Forward & Backward & Forward & Backward \\ \hline
                Base & \SI{6423000}{\nano\second} & \SI{17356000}{\nano\second} & \SI{72391743}{\nano\second} & \SI{165699492}{\nano\second} \\
                SIMD & \SI{2087000}{\nano\second} & \SI{10883000}{\nano\second} & \SI{31520540}{\nano\second} & \SI{112787278}{\nano\second}
            \end{tabular}
        \end{center}
        \caption{The runtimes of the second convolutional layer on Arm and x86, with and without SIMD.}
        \label{results:tab:runtimes_convolution}
    \end{table}

    \begin{table}[htb]
        \begin{center}
            \begin{tabular}{c|cccc}
                & \multicolumn{2}{c}{Arm} & \multicolumn{2}{c}{x86} \\ \hline
                & Forward & Backward & Forward & Backward \\ \hline
                Base & \SI{1265000}{\nano\second} & \SI{1774000}{\nano\second} & \SI{10867237}{\nano\second} & \SI{19407642}{\nano\second} \\
                SIMD & \SI{545000}{\nano\second} & \SI{457000}{\nano\second} & \SI{1787338}{\nano\second} & \SI{3546827}{\nano\second}
            \end{tabular}
        \end{center}
        \caption{The runtimes of the first linear layer on Arm and x86, with and without SIMD.}
        \label{results:tab:runtimes_linear}
    \end{table}

    \printbibliography[heading=bibintoc]
\end{document}